\documentclass[a4paper,9pt]{report}
\usepackage[T2A]{fontenc}
\usepackage[utf8]{inputenc}
\usepackage[russian,english]{babel}

\usepackage{PTSans}

\renewcommand*\familydefault{\sfdefault}

\usepackage[empty]{fullpage}
\usepackage{titlesec}
\usepackage{marvosym}
\usepackage[usenames,dvipsnames]{color}
\usepackage{verbatim}
\usepackage{enumitem}
\usepackage[hidelinks]{hyperref}
\usepackage{tabularx}
\usepackage{multicol}
\usepackage{graphicx}
\setlength{\multicolsep}{-3.0pt}
\setlength{\columnsep}{-1pt}
\RequirePackage{xcolor}
\RequirePackage{fontawesome5}

\addtolength{\oddsidemargin}{-0.6in}
\addtolength{\evensidemargin}{-0.5in}
\addtolength{\textwidth}{1.19in}
\addtolength{\topmargin}{-.7in}
\addtolength{\textheight}{1.4in}

\raggedbottom
\raggedright
\setlength{\tabcolsep}{0in}

\titleformat{\section}{
    \vspace{-4pt}\raggedright\large\bfseries
}{}{0em}{}[\color{YellowGreen}\titlerule \vspace{-5pt}]

\pdfgentounicode=1

\newcommand{\resumeItem}[1]{
    \item\small{
            {#1 \vspace{-2pt}}
    }
}

\newcommand{\classesList}[4]{
    \item\small{
            {#1 #2 #3 #4 \vspace{-2pt}}
    }
}

\newcommand{\resumeSubheading}[4]{
    \vspace{-2pt}\item
    \begin{tabular*}{1.0\textwidth}[t]{l@{\extracolsep{\fill}}r}
        \textbf{\normalsize#1} & \textbf{\small #2} \\
        \textit{\normalsize#3} & \textit{\small #4} \\

    \end{tabular*}\vspace{-7pt}
}

\newcommand{\resumeSubheadingAdditional}[2]{
    \vspace{-2pt}\item
    \begin{tabular*}{1.0\textwidth}[t]{l@{\extracolsep{\fill}}r}
        \normalsize#1 & \textbf{\small #2} \\

    \end{tabular*}\vspace{-7pt}
}

\newcommand{\resumeSubSubheading}[2]{
    \item
    \begin{tabular*}{0.97\textwidth}{l@{\extracolsep{\fill}}r}
        \textit{\small#1} & \textit{\small #2} \\
    \end{tabular*}\vspace{-7pt}
}

\newcommand{\resumeProjectHeading}[2]{
    \item
    \begin{tabular*}{1.001\textwidth}{l@{\extracolsep{\fill}}r}
        \small#1 & \textbf{\small#2}\\
    \end{tabular*}\vspace{-5pt}
}

\newcommand{\lineSeparatedText}[2]{
        {\textbf{\normalsize{{#1}}}} $|$ {\large{#2}}
}

\newcommand{\resumeExpItem}[2] {
    \item
    \begin{tabular*}{1.001\textwidth}{l@{\extracolsep{\fill}}r}
        #1 & \textbf{\small#2}
    \end{tabular*}\vspace{-5pt}
}

\newcommand{\resumeSubItem}[1]{\resumeItem{#1}\vspace{-4pt}}

\newcommand{\resumeSubHeadingListStart}{\begin{itemize}[leftmargin=0.0in, label={}]}
\newcommand{\resumeSubHeadingListEnd}{\end{itemize}}
\newcommand{\resumeItemListStart}{\begin{itemize}}
\newcommand{\resumeItemListEnd}{\end{itemize}\vspace{-5pt}}

\newcommand{\bulletItem}[1]{\item[$\bullet$] #1}

\begin{document}

\begin{center}
\textbf{\huge Корж Владислав} \\ \vspace{5pt}
\small
\textcolor{YellowGreen}{\faMobile} \hspace{.5pt} \href{tel:+79129150633}{+7 912 915 06 33}
$|$
\textcolor{YellowGreen}{\faAt} \hspace{.5pt} \href{mailto:korch937@mail.ru}{korch937@mail.ru}
$|$
\textcolor{YellowGreen}{\faTelegram} \hspace{.5pt} \href{https://t.me/solddddd}{solddddd}
$|$
\textcolor{YellowGreen}{\faLinkedin} \hspace{.5pt} \href{https://www.linkedin.com/in/sold666/}{sold666}
$|$
\textcolor{YellowGreen}{\faGithub} \hspace{.5pt} \href{https://github.com/sold666}{sold666}
$|$
\textcolor{YellowGreen}{\faMapMarker} \hspace{.5pt} \href{https://maps.app.goo.gl/VvvW63dZxLZH6iVR6}{Санкт-Петербург, Россия}
\end{center}

\section{Образование}
\resumeSubHeadingListStart
\resumeSubheading
{Санкт-Петербургский политехнический университет Петра Великого}{09.2020 -- 06.2024}
{Бакалавр, ИКНТ, "Программная инженерия", Средний балл: 4.8/5.0}{Санкт-Петербург, Россия}

\resumeSubheading
{НИУ ИТМО}{09.2024 -- ...}
{Магистр, ФИТиП, "Информационные системы бизнеса"}{Санкт-Петербург, Россия}
\resumeSubHeadingListEnd

\section{Дополнительное образование}
\resumeSubHeadingListStart
\resumeSubheadingAdditional
{Курс "Java-разработчик высоконагруженных приложений" \\ Образовательный центр VK в Политехе}{09.2022 -- 02.2023}
\resumeSubHeadingListEnd

\resumeSubHeadingListStart
\resumeSubheadingAdditional
{Курс "Web Development with Java Spring Framework" \\ Санкт-Петербургский политехнический университет Петра Великого}{10.2022}
\resumeSubHeadingListEnd

\resumeSubHeadingListStart
\resumeSubheadingAdditional
{Курс "The basics of Python programming" \\ LETPY}{09.2022 -- 02.2023}
\resumeSubHeadingListEnd

\section{Опыт}
\resumeSubHeadingListStart

\resumeExpItem {\lineSeparatedText{ПАО "Сбербанк" / Bell Integrator}{Java-разработчик}}{09.2024 -- ...}
\resumeItemListStart
\bulletItem{Я работаю внештатным сотрудником на проекте Sber Business Profile в Сбербанке.}
\bulletItem{SBP - это огромная единая база данных о клиентах Сбера и его Экосистемы.}
\resumeItemListEnd

\resumeExpItem {\lineSeparatedText{ООО "НПФ Беркут" (Bercut Ltd.)}{Java-разработчик}}{07.2023 -- 08.2024}
\resumeItemListStart
\bulletItem{Работа в отделе автоматизации тестирования включает разработку End-to-End тестов для биллинговой системы, обслуживающую одного из ведущих операторов сотовой связи страны}
\bulletItem{Рефакторинг кодовой базы тестирующего фреймворка, разработка Unit-тестов, анализ Allure отчетов. Работа с БД PostgreSQl, сборками в Jenkins}
\bulletItem{Реализация maven-библиотеки "LogRunMan" для сбора информации выполнения автотестов}
\bulletItem{Использование протокола SOAP, REST API}
\resumeItemListEnd

\resumeExpItem {\lineSeparatedText{ООО "Среда" (Sreda Solutions)}{Программист-стажер}}{06.2023 -- 07.2023}
\resumeItemListStart
\bulletItem{Участие в разработке backend-части программного обеспечения, основанного на технологии V2x}
\bulletItem{Разработка Unit-тестов, написание документации}
\bulletItem{Реализация bash-скрипта с использованием API GitLab для сбора проектной информации}
\resumeItemListEnd

\resumeExpItem {\lineSeparatedText{ООО "Санкт-Петербургский центр разработок EMC" (Dell EMC)}{Практикант}}{02.2022 -- 01.2023}
\resumeItemListStart
\bulletItem{Интеграция, конфигурация и мониторинг Bare-metal CSI Driver с помощью технологий Prometheus, Grafana}
\bulletItem{Изучение работы с K8s, Docker}
\bulletItem{Работа с bash-скриптами}
\resumeItemListEnd

% \resumeExpItem {\lineSeparatedText{Resume-Bot}{Java-разработчик}}{09.2023 -- 06.2024}
% \resumeItemListStart
% \bulletItem{Написание всей клиентской части бота (команды, состояния, диалог с пользователем, логика записи информации в БД)}
% \bulletItem{Написание FS, HLD документации}
% \bulletItem{Настройка GitHub CI/CD}
% \bulletItem{Разработка Unit-тестов, интеграционных тестов, работа с HH API и LaTeX}
% \resumeItemListEnd

\resumeSubHeadingListEnd

\vspace{-10pt}

\section{Навыки}
\begin{itemize}[leftmargin=0.15in, label={}]
\small{\item{

\textbf{Языки программирования:}{ Java, SQL, Python, C++, JavaScript, Bash} \\ \vspace{3pt}

\textbf{Технологии:}{ Spring Boot, PostgreSQl, JUnit, Jenkins, Git, SVN, Apache Maven, Docker, K8s, Kafka, Prometheus, Grafana} \\ \vspace{3pt}

\textbf{Инструменты:}{ IntelliJ IDEA, GitHub, GitLab, Postman, Allure, SoapUI, Jira, Trello, Kaiten, Slack} \\ \vspace{3pt}

}}
\end{itemize}
\vspace{-10pt}

\section{Дополнительно}

\begin{itemize}
\bulletItem{Опыт написания модульных и интеграционных тестов с использованием Mockito, Hamcrest, Surefire}
\bulletItem{Опыт написания документации с использованием Markdown, AsciiDoc и LaTeX}
\bulletItem{Опыт разработки и ведения команды Android-приложения (Kotlin, Gradle, Material Design, Firebase, Figma, Coroutines, Retrofit, Jetpack Navigation, Picasso)}
%\begin{itemize}
%\bulletItem Руководство командой разработки: планирование, координация и контроль выполнения задач
%\bulletItem Продумывание логики приложения: архитектура, функциональные требования, оптимизация процессов
%\bulletItem Реализация дизайн-составляющей: взаимодействие с дизайнером из OK, внедрение UI/UX решений, обеспечение соответствия дизайна техническим требованиям
%\end{itemize}
\bulletItem{Опыт разработки и верстки фронтенд-части веб-приложений (Node.js, Figma, HTML, CSS, Parcel, Chart.js)}
\bulletItem{Опыт разработки десктопных приложений (Java, JavaFX, JDBC, Microsoft SQL Server)}
\bulletItem{Опыт разработки телеграм-ботов (Java, Spring Boot, PostgreSQl, интеграция с сторонним API, Emoji Java, TelegramBots, Jacoco)}
\bulletItem{Знание английского языка: Intermediate}
\end{itemize}

\end{document}
